\documentclass[12pt]{article}
% We can write notes using the percent symbol!
% The first line above is to announce we are beginning a document, an article in this case, and we want the default font size to be 12pt
\usepackage[utf8]{inputenc}
% This is a package to accept utf8 input.  I normally do not use it in my documents, but it was here by default in Overleaf.
\usepackage{pgfplots}
\usepackage{amsmath}
\usepackage{amssymb}
\usepackage{amsthm}
% These three packages are from the American Mathematical Society and includes all of the important symbols and operations 
\usepackage{fullpage}
% By default, an article has some vary large margins to fit the smaller page format.  This allows us to use more standard margins.

\setlength{\parskip}{1em}
% This gives us a full line break when we write a new paragraph

\begin{document}
% Once we have all of our packages and setting announced, we need to begin our document.  You will notice that at the end of the writing there is an end document statements.  Many options use this begin and end syntax.

\begin{center}
    \large  Learning mathematics from S.N Ram
\end{center}
\section{optimization}
A matrix X is called positive semidefinite if it is symmetric and all its eigenvalues are non-negative. If all eigenvalues are strictly positive then it is called a positive definite matrix\\

Minimize $C.X( i.e\sum_{i}\sum_{j}C_{ij}X_{ij}$ )    S.T $A_{i}.X=b_{i}$ and $X>0$, If X is symmetric, C is also symmetric and further the optimization form is of SDP(semi definite programming form).Notice that SDP looks remarkably similar to a linear program
It is easy to see that a linear program LP is a special instance of an SDP by
\[A_{i}=\begin{pmatrix} 
a_{11} & 0 & 0.&.&.& \\0 &  a_{22}&  0.&.&.&\\0 & 0 & a_{33}.&.&.&\\. &..&. &. &.\end{pmatrix}
\]$i=1,2,3... $and \[C_{i}=\begin{pmatrix} c_{11} & 0 & 0.&.&.& \\0 &  c_{22}&  0.&.&.&\\0 & 0 & c_{33}.&.&.&\\. &..&. &. &.\end{pmatrix}\]
$LP=SDP=Min \ C.X \  S.T \ A.X=b_{i}\ and\ X>0$ which implies $X =\begin{pmatrix} x_{11} & 0 & 0.&.&.& \\0 &  x_{22}&  0.&.&.&\\0 & 0 & x_{33}.&.&.&\\. &..&. &. &.\end{pmatrix}$
\subsection{Applications of  SDP}
 In the SDP framework include:linear inequalities, convex quadratic inequalities, lower bounds on matrix norms, lower bounds on determinants of symmetric positive semidefinite
matrices, lower bounds on the geometric mean of a nonnegative vector, plus
many others.
MAX CUT Problem:\\


\subsection{How to solve SDP}
There is no finite algorithm for solving SDP. There is a simplex algorithm, but it is not a finite algorithm. There is no direct analog of a “basic feasible solution” for SDP
\subsection{Eigen value magics}
\subsubsection{Nodal voltage using Eigen values}
Calculating nodal voltages and branch current flows in a meshed network using Eigen values and eigen vectors of Laplacian matrix ($Y_{bus}$)
\begin{center}
$I=YV$ \\  $I=PDP^{T}V$  (since $Y_{bus}$ is symmetric matrix ) \\ $P^{T}I=DP^{T}V$\\  \end{center}
\[\begin{pmatrix} 
\sum_{k=1}^N1^TI_{k} \\
\sum_{k=1}^Nu_{k2}I_{k} \\
\sum_{k=1}^Nu_{k3}I_{k} \\
...\\
..\\\end{pmatrix} =\begin{pmatrix} 0 \\ \lambda_{2}\sum_{k=1}^Nu_{k2}V_{k} \\ \lambda_{3}\sum_{k=1}^Nu_{k3}V_{k} \\...\\..\\\end{pmatrix}\] since, $\lambda_1=0 \ and\  its \ eigen\ vector\ is \ 1$

\[\begin{pmatrix} 
\sum_{k=1}^N1^TI_{k} \\
\sum_{k=1}^Nu_{k2}I_{k} \\
\sum_{k=1}^Nu_{k3}I_{k} \\
...\\
..\\\end{pmatrix} =\begin{pmatrix}1 & 0 \\ 0 & diag(\lambda)\\ \end{pmatrix} \begin{pmatrix}1^T  \\ U \\ \end{pmatrix} V\]

\begin{center} $V_m=\sum_{k=2}^N u_{mk}(b_{k}/\lambda_{k}) $  as $ b_k=\sum_{j=2}^N u_{jk}I_{j}$ \end{center}

\section{Probability}

The corner stone of network reduction initiated  by Ward [] with the introduction of circuit equivalents using Kron reduction, where mainly focussed on retaining the effect of external circuits on boundary buses
\[Y
=
\begin{pmatrix} 
Y_{EE} & Y_{EB} & 0 \\
Y_{BE} &  Y_{BB}&  Y_{BI}\\
0 & Y_{IB} & Y_{II}
\end{pmatrix}
as \ Y_{EB}=Y_{BE} ;\ Y_{BI}=Y_{IB}
\]

\[Y_{red}
=Y-1/Y_{EE}
\begin{pmatrix} 
Y_{EE}  \\
Y_{BE}\\
0
\end{pmatrix}
\begin{pmatrix} 
Y_{EE}  \\
Y_{EB}\\
0
\end{pmatrix}^T
=
\begin{pmatrix} 
0 & 0 & 0 \\
0 &  Y_{BB}-(Y_{BE}*Y_{EB})/Y_{EE} &  Y_{BI}\\
0 & Y_{IB} & Y_{II}
\end{pmatrix}
\]

\[P_{red}
=P-1/Y_{EE}
\begin{pmatrix} 
Y_{EE}  \\
Y_{BE}\\
0
\end{pmatrix}
P_{EE}
=
\begin{pmatrix}
P_{EE}-P_{EE}\\
P_{BB}-Y_{BE}*P_{EE}/Y_{EE}\\
P_{II}
\end{pmatrix}
\]
However the above method lacks the retainment of system as per the geographical location(zones) and it is more of attention network i.e it equilizes all the buses which are outside of the required zone and the result are dependant on area specific.

New method, called Spectral Clustering proposed for Network reduction based on geographic (nearest neighbor) diatance ,i.e, Clsutering busees which are closes to each other by measuring the distance. Using Gaussian similarity metric, as it gives non linear and normalization function for euclidean distance  (0 and 1)  $S_{ij}=e^{(-W_{ij}||X_{i}-X_{j}||)/\sigma)}$ here $\sigma$ controls how spread the neighbours are and $Y=D^1-W$

$X^TSX=\sum_{ij}A_{ij}(X_{i}-X_{j})^2$ where $S=D-A$
Eigen vector corresponding to 2nd least eigen value, gives the bipartie clustering information ( two groups), finding K means from K eigen vectors corresponding to K lowest eigen value gives K clusters which formed by gausiian similar matrx.
\begin{center}
Min $X^TSX$ \\
S.T $X^TX=1$ 
\end{center}
The major drawback of the spectral clustering is, it preserves the local informations, but lacks the inter cluster information (tie line connections).



Further, the development of idea of Network reduction based on PTDF(Power transfer distribution Factor) based comarasion technique.i.e Original network is divided into zones and by minimizing the error between the Zonal PTDF value with reduced network gives the equilant reduced network which arguises the application on online security and market related planning studies.

$P_{Lineflows}=H_{PTDF}P_{Bus _inj}$  and$ P_{zonal flow}=T_{z}H_{PTDF}P_{Bus _inj}$ caused $P_{red}=H_{red PTDF}T_{bz}P_{Bus _inj}$.
$T_{bZ}$ matrix be obtained from Zonal information data (Bus aggregation method) and $T_{z}$ is inter zonal power flow (tie line flow) may be obtained from data(i.e summing the  inter zonal power between the aggregated buses)
$H_{redPTDF}$ be obtained by minimizing the nter zonal flows of original network to reduced network  flows 
\begin{center}
Min  $||P_{red}-P_{zonalflow}||^2$\\
\end{center}

 Further, $H_{redPTDF}$ can be represented as $Y_{newBR}Y_{new}^{-1}$. As $Y_{newBR}=diag(1/imp_{new})inc_{new}^T$ and $Y_{new}^{-1}=inc_{new}^{-T}diag(imp_{new})inc_{new}^{-1}$

Which implies  $H_{redptdf}inc_{new}=I$

The basic problem with above method is poorly estimating $Y_{new}$ from $H_{redPTDF}$. 

The proposed method slights modify the PTDF reduction method by throwing the laplacian contraints.

\section{idea}

For a given Network $Y_{bus}$ matrix, using Spectral clustering finding the tie line connected nodes inbetween the clusters. By using Kron reduction(Gauss elimination of nodes) finding the equillant network by removing the clustered nodes leads to the network consists of only Tie line connected nodes and its inter connectivity.

\section{notes}

Also note that for invertible A,$det(A)=$det(A$^{-1}$)$^{-1}$ and ,$det(A)$det(A$^{-1}$)=1 If S is invertible, 
then det(SAS$^{-1}$) =$det(A)$ . Hence similar matrices have the same determinant.Thus it makes sense to define the determinant det T of a linear transformation (i.e., this determinant

\subsection{Inverse of the Covariance matrix }

$L=Adj-D=\sum^{-1}$ i.e Inverse of variance-covariance matrix of Gausian distribution give Laplace matrix.
Where L is semi-definite and 1 is always an eigenvector for the eigenvalue 0
A spanning tree is a subset of Graph G, which has all the vertices covered with minimum possible number of edges. Hence, a spanning tree does not have cycles and it cannot be disconnected.

No of spanning trees $=1/n(\lambda_1\lambda_2...\lambda_{n-1})$

An eigenvector in general represents a natural mode a system can take. In this case, the system is a graph. And eigenvectors represent these modes
the eigenvectors kind of represent the 'frequency' present in a particular weighted graph. They form the x axis if you plot a frequency spectrum.
The eigenvalues in turn, represent the 'amounts' of these eigenvectors present in a given graph signal. They form the y axis of the frequency spectrum

Let $X \hookrightarrow  \mathcal{N}(\mu,\,\sigma^{2})$ for some $\mu  \in \mathcal{R}^d$ and $\sum \in \mathcal{S}^d$. Then , there exists a matrix  $B\in \mathcal{R}^{dxd}$ such that if we define $Z=B^{-1}(X-\mu)$, then $Z\hookrightarrow\mathcal{N}(0,1)$

The central limit theorem states that the distribution of sample means approximates a normal distribution as the sample size gets larger (assuming that all samples are identical in size), regardless of population distribution shape.

Suppose $P_{inj} \sim {\cal N}(\mu,\sigma^2)$ and and PTDF matrix is linear matrix which leads to the $P_{line} \sim{\cal N}(\mu_{new},\sigma_{new}^2)$  where $ \mu_new=ptdf*\mu$ and $\sigma_{new}=ptdf*\sigma$

\section{Idea of mimicing}
\subsection{Inverse laplacian}

Our goal is to transfer the original space $B_{orig}$  into latent space( reduced) $B_{new}$. Since $Y_{bus}$ is sparese and inverse $Y_{bus}$ after removing Kth row  and Kth column is full matrix. 

As we know that $ B_{orig}=L$(Laplacian matrix)
where $ L^{-1}=B_{orig}^{-1}=\sum$(variance co variance of bus nodes), here $\sum$ gives , how two nodes are related.

By using PCA (Prinicple component analysisi) we project the $\sum$ into eigen sapce where theore variance is maximum

$\sum=R_{Bus,NO of features} R_{Bus,NO of features}^T$

By matrix factorization

$R_{Bus, no of features}=P_{bus,N.latent bus}W_{N.latent bus, No of latent bus}$

$B_{new}=W_{N. latent bus, N.latent bus}^{-T} W_{N.latent bus, N.latent bus}^{-1}=W_{N.latent bus, N.latent bus}^{-2}$

$B=\phi B_{new} \phi^1$ where $\phi=transforming matrix from orig to lowdimen and  \phi^1$ is vice versa

$L=D-A$

$L^{-1}=\sum$

$\sum=(D-A)^-1 = D^{-1}+D^{-1}(I-AD^{-1})^{-1}AD^{-1}$ 

$\sum=D^{-1}PD^{-1}$

AS $P=(I+(I-AD^{-1})^{-1}A)$

$\frac{\partial \sum}{\partial P}=D^{-1}\frac{\partial P}{\partial A}D^{-1}=D^{-1}((I-AD^{-1})^{-1}-A(I-AD^{-1})^{-1}D^{-1}(I-AD^{-1})^{-1})D^{-1}$

let $D=I$

$\frac{\partial P}{\partial A}=(I-A)^{-1}-A(I-A)^{-1}(I-A)^{-1}=(I-A)^{-1}=P$

$A=log(P)$



\begin{center}


Min $||W_1B_{new}^{-1}W_2P_{bus}-\theta||^2$\\

S.T $W_1ij>0$ and $W_2ij>0$ \\
 projection matrix should be positive values only

Further, $W_2P_{bus}-P_{new}$=$0$ \\
i.e  projection matrix from original $p_{bus} $ to latent space $p_{new}$\\




\end{center}

Backpropogation idea envisaged to solve the above optimization problem.

\begin{center}
$Error$=$(W_1B_{new}^{-1}W_2P_{bus}-\theta)$

$\frac{\partial L}{\partial W_1}$=$Error(B_{new}^{-1}W_2P_{bus})^T$


$\frac{\partial L}{\partial W_2}$=$B_{new}^{-1}W_1{^T}ErrorP^T$
\end{center}




\end{document}